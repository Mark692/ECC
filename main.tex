\documentclass[a4paper,12pt]{tesiinfo}
%\documentclass[a4paper,12pt,dvipdfm]{tesiinfo}

%%Pacchetti utili anche se non necessari
\usepackage{amsfonts}
\usepackage{amsmath}
\usepackage{latexsym}
\usepackage{tabularx}
\usepackage[italian]{babel}
\usepackage[bookmarks=true]{hyperref}
\usepackage{subfigure}
\usepackage{graphicx}
\usepackage{amssymb}



\titolo{Crittografia Ellittica}
\laureando{Marco Carolla}
\relatore{Gabriele Di Stefano}
\annoaccademico{2016-2017}

\begin{document}

\maketitle
\contentspage
\chapter{Introduzione}


%\textbf{Teoria dei Gruppi}: branca della matematica e dell'algebra astratta che studia le strutture algebriche conosciute come \textit{Gruppi}. 
%\newline\newline
%
%
%
\textbf{Gruppo}: insieme $G$ ed un'operazione $\bullet$ detta \textit{Operazione o Legge del gruppo G} che combina due elementi qualsiasi $a$ e $b$ dell'insieme $G$ formandone un terzo $c = a\bullet b$ appartenente ancora allo stesso insieme $G$. 
La coppia $(G, \bullet)$ deve soddisfare gli \textit{assiomi di gruppo}:
\begin{enumerate}
    \item \textit{Chiusura}: $ c \in G$, dove $c = a\bullet b$ $ \forall (a$, $b) \in G$;
    %
    %
    \item \textit{Associativit\`a}: $\left ( a\bullet b \right ) \bullet c = a \bullet \left ( b \bullet c \right ) $, $ \forall (a, b, c) \in G$;
    %
    %
    \item \textit{Elemento Identit\`a}: $ \exists ! e \in G \mid e \bullet a = a \bullet e = a$;
    %
    %
    \item \textit{Elemento Inverso}: $\exists b \in G \mid a \bullet b = b \bullet a = e$ dove $b = a^{-1} $ se l'operazione \`e definita come $"\cdot"$ oppure $b = -a $ se l'operazione \`e definita come $"+"$, infine l'elemento $e$ \`e l'identit\`a.
\end{enumerate}
Un esempio concreto di Gruppo viene dato dai \textit{Gruppi di Permutazione}. Questi sono definiti da un insieme $X$ ed una collezione $G$ delle corrispondenze biunivoche di $X$ su se stesso, anche dette Permutazioni. Il cubo di Rubik \`e un elemento di questo gruppo ed ogni rotazione di una sua superficie risulta in una permutazione dei colori sulla superficie. 
\newline
L'operazione di gruppo \`e chiamata Composizione di permutazioni ed \`e definita come prodotto di due permutazioni $\sigma$ e $\pi$ tali per cui la loro composizione $\sigma \cdot \pi$ mappa ogni elemento $x$ all'insieme $\sigma ( \pi ( x ) )$. Essendo vero che la composizione di due corrispondenze biunivoche \`e ancora una corrispondenza biunivoca, il prodotto di due permutazioni \`e ancora una permutazione; inoltre la composizione \`e associativa e cos\`i anche il prodotto di permutazione $ (\sigma \cdot \pi) \cdot \rho = \sigma \cdot ( \pi \cdot \rho )$.
\newline
L'elemento neutro per l'operazione di Composizione \`e la Permutazione Identit\`a che mappa ogni elemento dell'insieme su se stesso. 
\newline
Infine, le corrispondenze biunivoche sono di inversa, cos\`i quindi anche le permutazioni e l'inversa di $\sigma$ ovvero $ \sigma ^ {-1} $ \`e ancora una permutazione.
\newline
Avendo verificato che esiste un prodotto associativo, un elemento identit\`a ed un'inversa per ogni suo elemento, l'insieme delle permutazioni forma un gruppo.
%
%
%
\newline\newline
A questi assiomi possiamo aggiungere la commutativit\`a, ovvero il risultato dell'operazione di gruppo pu\`o dipendere dall'ordine degli operandi per cui $ a \bullet b = b \bullet a$ pu\`o non essere sempre vera. 
\newline
I gruppi per i quali vale la commutatività sono detti \textit{Gruppi Abeliani}.
\newline\newline
%
%
%
\textbf{Anello}: Un anello \`e un insieme sul quale sono definite due operazioni binarie (struttura algebrica) che, per estensione dei numeri interi $\mathbb{Z}$, vengono chiamate \textit{somma} ($+$) e \textit{prodotto} ($\cdot$). L'anello \`e un gruppo abeliano sull'operazione binaria della (somma); consideriamo un esempio sui numeri interi: l'insieme \`e $\mathbb{Z}$, l'operazione \`e la somma "+" ed il gruppo viene quindi denotato con $(\mathbb{Z}, +)$. L'operazione + combina due elementi qualsiasi e ne forma un terzo, \`e un'operazione associativa, l'elemento identit\`a \`e lo zero, ogni intero $n \in \mathbb{Z}$ ha il suo elemento opposto $-n$, infine \`e un'operazione commutativa dato che vale $m + n = n + m, \forall (n, m) \in \mathbb{Z}$.
\newline
I numeri interi $\mathbb{Z}$ presi in modulo $n$ costituiscono il gruppo $\mathbb{Z}/ n \mathbb{Z}$ degli \textit{Interi in modulo $n$}. Tali gruppi $G$, detti ciclici, possono esser generati da un unico elemento, consistono in un insieme di elementi con una singola operazione associativa invertibile e contiene un elemento $g$, detto \textit{Generatore} del gruppo, tale che ogni altro elemento del gruppo pu\`o essere ottenuto applicando ripetitivamente l'operazione di gruppo o la sua inversa a tale elemento $g$. Ogni elemento del gruppo pu\`o esser scritto come potenza di $g$ nelle notazioni della moltiplicazione o dell'addizione.
\newline
Per un anello pu\`o valere l'uguaglianza $0 = 1$ qualora sia formato da un solo elemento; viene perci\`o detto \textit{anello banale}.
\newline\newline
%
%
%
\textbf{Caratteristica di un Anello}: Dato un anello $R$ ed un elemento $a$ la sua caratteristica $char(R)$ \`e definita come \begin{center} $n  \in \mathbb{N}$, $n \ne 0 $, $char(R) = \begin{cases} \min (n) \mid \underbrace{a+a+\cdots+a}_\text{n volte} = 0 & \mbox{se }\exists n\\
0 & \mbox{altrimenti}
\end{cases}$
\end{center}
Se tale minimo $n$ non esiste allora si assume che la caratteristica sia 0 per definizione.
Per un anello la caratteristica diventa il pi\`u piccolo $n$ tale che $a+a+\cdots+a{{=}}0$ per ogni suo elemento; la si pu\`o definire come il \textit{minimo comune multiplo} delle caratteristiche di tutti i suoi elementi. 
Tutti gli anelli con caratteristica $0$ sono \textit{infiniti}.\newline
L'unico anello con caratteristica $1$ \`e l'anello banale, formato da un solo elemento, per il quale vale la seguente $0 {{=}} 1$.\newline
Tutti gli altri anelli con caratteristica maggiore di 1 sono costituiti da un numero finito di elementi. Suddivisioni dei numeri interi del tipo $\mathbb{Z}/n\mathbb{Z}$ forniscono anelli modulari con elementi in numero finito ed hanno caratteristica $n$.\newline
I campi $\mathbb{Q}$, $\mathbb{R}$ e $\mathbb{C}$ hanno caratteristica $0$.
\newline\newline
%
%
%
\textbf{Spazio Proiettivo}: \`e ottenuto da uno spazio euclideo aggiungendo i \textit{punti all'infinito}. A seconda dello spazio euclideo considerato, ad esempio una retta o un piano, si vengono a definire, nello spazio proiettivo, una retta proiettiva o un piano proiettivo. 
Estendere il piano Euclideo a quello proiettivo significa:
\begin{enumerate}
    \item per ogni classe di rette parallele, aggiungere un singolo nuovo punto. Tale punto \`e considerato come punto di incontro per ogni retta della stessa classe. Classi di diverse rette parallele avranno diversi punti. Questi punti sono chiamati \textit{punti all'infinito};
    \item aggiungere una nuova retta incidente su tutti e soli i punti all'infinito. Questa retta \`e chiamata \textbf{la} \textit{retta all'infinito}.
\end{enumerate}
Quanto appena detto esclude l'esistenza di rette parallele.\newline
Lo spazio proiettivo $n$-dimensionale \`e l'unione di $\mathbb{R}^n$ e di tutti i suoi punti all'infinito. Tuttavia questa definizione pone i punti all'infinito come \textit{punti speciali}, per tal motivo una definizione pi\`u rigorosa pu\`o essere: Lo spazio proiettivo $n$-dimensionale \`e definito come l'insieme delle rette $\mathbb{R}^ {n+1}$ passanti per l'origine.\newline
Lo spazio proiettivo reale $\textbf{RP}^n$ o $\mathbb{P}_n(\mathbb{R})$ \`e lo spazio topologico di rette di $\mathbb{R}^ {n+1}$ passanti per l'origine.
\newline\newline
%
%
%
\textbf{Variet\`a - \textit{Manifold}}: Una variet\`a algebrica \`e l'insieme delle soluzioni di un sistema di polinomi sui numeri reali o complessi. In una variet\`a algebrica V chiamiamo il punto P \textit{singolare} in senso geometrico se lo spazio tangente a V nel punto P non pu\`o essere definito regolarmente. Per variet\`a definite sui reali il concetto di punto singolare generalizza il concetto di \textit{non derivabilit\`a}. Una variet\`a algebrica che non presenti punti singolari viene detta non singolare, \textit{liscia} o $\mathbb{C}^ {\infty}$ ed implica l'esistenza di derivate di ogni ordine.
Classi di variet\`a algebriche sono le curve algebriche piane, le quali includono: rette, circonferenze, parabole, ellissi, iperbole, curve cubiche come ad esempio le curve ellittiche.
Un punto del piano appartiene alla curva se le sue coordinate soddisfano l'equazione polinomiale data di cui la variet\`a algebrica offre la soluzione.
\newline\newline
%
%
%
\textbf{Curva piana cubica}: curva algebrica piana $C$ definita da un'equazione cubica $F(x, y, z) = 0$ applicata su di un piano proiettivo. Combinazioni lineari non nulle di monomi del terzo grado sono: $x^3, y^3, z^3, x^2y, x^2z, y^2x, y^2z, z^2x, z^2y, xyz$, per un totale di 10 combinazioni. Per questo motivo le curve cubiche formano uno spazio proiettivo di dimensione 9 per ogni campo $K$ dato.
Una curva $C$ passante per $P$ comporta che ogni punto $P$ impone una singola condizione lineare sulla $F$; per tal motivo \`e possibile trovare curve cubiche su qualsiasi 9 punti dati
\newline\newline
%
%
%
\textbf{Genere di una curva}
\newline\newline
%
%
%
\textbf{Punto Razionale}
\newline\newline
%
%
%
\textbf{Curva Cubica}
\newline\newline
%
%
%
\textbf{Curva Non Singolare}
\newline\newline
%
%
%
\textbf{Funzione Abeliana}: funzione $f(u_1, ... u_p)$ analitica uniforme nelle variabili p, periodica, dipendente da tutte e p le sue variabili, meromorfa.
\newline\newline
%
%
%
\textbf{Funzione Meromorfa}: \`e una funzione Olomorfa (Derivabile nel campo complesso per ogni direzione) in un sottodominio D del campo complesso $\mathbb{C}$, ad esclusione di un numero finito di punti detti Poli della funzione\newline\newline
%
%
%
%
%
%
%
%
%
%
\chapter{Curve Ellittiche}

Una curva ellittica pu\`o esser definita come una Curva Algebrica, di Genere 1, avente un Punto Razionale.

Un modello atto ad esprimere una curva ellittica \`e una Curva Cubica, Non Singolare. Tale modello presenta un punto di flesso all'infinito ed equivale a dire che la curva pu\`o esser scritta nella forma estesa di Tate-Weierstrass ovvero:

$y^{2}z + a_1xyz + a_3yz^2 =x^3 + a_2x^2z + a_4xz^2 + a_6z^3$
%
%
%
%
%
%
%
%
%
%
\bibliografia{tesi}
%
%
%
%
%
%
%
%
%
%
\appendice
\chapter{prima appendice}
%
%
%
%
%
%
%
%
%
%
\chapter{seconda appendice}
%
%
%
%
%
%
%
%
%
%
\end{document}
