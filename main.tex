\documentclass[a4paper,12pt]{tesiinfo}
%\documentclass[a4paper,12pt,dvipdfm]{tesiinfo}

%%Pacchetti utili anche se non necessari
\usepackage{amsfonts}
\usepackage{amsmath}
\usepackage{latexsym}
\usepackage{tabularx}
\usepackage[italian]{babel}
\usepackage[bookmarks=true]{hyperref}
\usepackage{subfigure}
\usepackage{graphicx}



\titolo{Crittografia Ellittica}
\laureando{Marco Carolla}
\relatore{Gabriele Di Stefano}
\annoaccademico{2016-2017}

\begin{document}

\maketitle
\contentspage
\chapter{Introduzione}


\textbf{Gruppo}: insieme $G$ ed un'operazione $\bullet$ detta \textit{Legge del gruppo G} che combina due elementi qualsiasi $a$ e $b$ formandone un terzo $a\bullet b$ appartenente ancora allo stesso insieme $G$. 
La coppia $(G, \bullet)$ deve soddisfare gli \textit{assiomi di gruppo}:
\begin{itemize}
    \item Chiusura: per ogni coppia $a$ e $b$, il risultato dell'operazione $a\bullet b$ \`e ancora in $G$;
    \item Associativit\`a: per ogni $a$, $b$ e $c$ in $G$ vale $\left ( a\bullet b \right ) \bullet c = a \bullet \left ( b \bullet c \right ) $;
    \item Elemento Identit\`a: esiste un unico elemento $e$ in $G$ tale che sia verificata l'equazione $ e \bullet a = a \bullet e = a$;
    \item Elemento Inverso: per ogni elemento $a$ in $G$ esiste un elemento $b$ di $G$ indicato come $a^{-1}$ (o come $-a$ se l'operazione \`e definita come $"+"$) tale per cui sia verificata l'equazione $ a \bullet b = b \bullet a = e$, dove $e$ \`e l'elemento identit\`a.
\end{itemize}
Inoltre il risultato dell'operazione pu\`o dipendere dall'ordine degli operandi per cui l'equazione $ a \bullet b = b \bullet a$ pu\`o non essere sempre vera. I gruppi per i quali vale la commutatività sono detti \textit{Gruppi Abeliani}.
\newline\newline
\textbf{Gruppo Abeliano}: gruppo per il quale valgono i quattro assiomi dei Gruppi ed un quinto assioma detto \textit{Commutativit\`a} per il quale vale la propriet\`a commutativa dei suoi elementi. Un gruppo abeliano \`e anche detto \textit{gruppo commutativo}.
\newline\newline
\textbf{Spazio Proiettivo}: \`e ottenuto da uno spazio euclideo aggiungendo i \textit{punti all'infinito}. A seconda dello spazio euclideo considerato, ad esempio una retta o un piano, si vengono a definire nello spazio proiettivo, una retta proiettiva o un piano proiettivo.
\newline\newline
\textbf{Funzione Algebrica}
\newline\newline
\textbf{Genere di una curva}
\newline\newline
\textbf{Punto Razionale}
\newline\newline
\textbf{Curva Cubica}
\newline\newline
\textbf{Curva Non Singolare}
\newline\newline
\textbf{Funzione Abeliana}: funzione $f(u_1, ... u_p)$ analitica uniforme nelle variabili p, periodica, dipendente da tutte e p le sue variabili, meromorfa.\newline\newline
\textbf{Funzione Meromorfa}: \`e una funzione Olomorfa (Derivabile nel campo complesso per ogni direzione) in un sottodominio D del campo complesso $\mathbb{C}$, ad esclusione di un numero finito di punti detti Poli della funzione\newline\newline
















\chapter{Curve Ellittiche}

Una curva ellittica pu\`o esser definita come una Curva Algebrica, di Genere 1, avente un Punto Razionale.

Un modello atto ad esprimere una curva ellittica \`e una Curva Cubica, Non Singolare. Tale modello presenta un punto di flesso all'infinito ed equivale a dire che la curva pu\`o esser scritta nella forma estesa di Tate-Weierstrass ovvero:

$y^{2}z + a_1xyz + a_3yz^2 =x^3 + a_2x^2z + a_4xz^2 + a_6z^3$



\bibliografia{tesi}

\appendice
\chapter{prima appendice}
\chapter{seconda appendice}

\end{document}
