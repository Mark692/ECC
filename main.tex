\documentclass[a4paper,12pt]{tesiinfo}
%\documentclass[a4paper,12pt,dvipdfm]{tesiinfo}

%%Pacchetti utili anche se non necessari
\usepackage{amsfonts}
\usepackage{amsmath}
\usepackage{latexsym}
\usepackage{tabularx}
\usepackage[italian]{babel}
\usepackage[bookmarks=true]{hyperref}
\usepackage{subfigure}
\usepackage{graphicx}
\usepackage{amssymb}
\graphicspath{ {Images/} }



\titolo{Crittografia Ellittica}
\laureando{Marco Carolla}
\relatore{Gabriele Di Stefano}
\annoaccademico{2016-2017}

\begin{document}

\maketitle
\contentspage
\chapter{Introduzione}


%\textbf{Teoria dei Gruppi}: branca della matematica e dell'algebra astratta che studia le strutture algebriche conosciute come \textit{Gruppi}. 
%\newline\newline
%
%
%
\textbf{Gruppo}: insieme $G$ ed un'operazione $\bullet$ detta \textit{Operazione o Legge del gruppo G} che combina due elementi qualsiasi $a$ e $b$ dell'insieme $G$ formandone un terzo $c = a\bullet b$ appartenente ancora allo stesso insieme $G$. 
La coppia $(G, \bullet)$ deve soddisfare gli \textit{assiomi di gruppo}:
\begin{enumerate}
    \item \textit{Chiusura}: $ c \in G$, dove $c = a\bullet b$ $ \forall (a$, $b) \in G$;
    %
    %
    \item \textit{Associativit\`a}: $\left ( a\bullet b \right ) \bullet c = a \bullet \left ( b \bullet c \right ) $, $ \forall (a, b, c) \in G$;
    %
    %
    \item \textit{Elemento Identit\`a}: $ \exists ! e \in G \mid e \bullet a = a \bullet e = a$;
    %
    %
    \item \textit{Elemento Inverso}: $\exists b \in G \mid a \bullet b = b \bullet a = e$ dove $b = a^{-1} $ se l'operazione \`e definita come $"\cdot"$ oppure $b = -a $ se l'operazione \`e definita come $"+"$, infine l'elemento $e$ \`e l'identit\`a.
\end{enumerate} 
Un esempio concreto di Gruppo viene dato dai \textit{Gruppi di Permutazione}. Questi sono definiti da un insieme $X$ ed una collezione $G$ delle corrispondenze biunivoche di $X$ su se stesso, anche dette Permutazioni. Il cubo di Rubik \`e un elemento di questo gruppo ed ogni rotazione di una sua superficie risulta in una permutazione dei colori sulla superficie. 
\newline
L'operazione di gruppo \`e chiamata Composizione di permutazioni ed \`e definita come prodotto di due permutazioni $\sigma$ e $\pi$ tali per cui la loro composizione $\sigma \cdot \pi$ mappa ogni elemento $x$ all'insieme $\sigma ( \pi ( x ) )$. Essendo vero che la composizione di due corrispondenze biunivoche \`e ancora una corrispondenza biunivoca, il prodotto di due permutazioni \`e ancora una permutazione; inoltre la composizione \`e associativa e cos\`i anche il prodotto tra permutazioni $ (\sigma \cdot \pi) \cdot \rho = \sigma \cdot ( \pi \cdot \rho )$.
\newline
L'elemento neutro per l'operazione di Composizione \`e la Permutazione Identit\`a che mappa ogni elemento dell'insieme su se stesso. 
\newline
Esistendo l'inversa di una corrispondenza biunivoca, esister\`a l'inversa $ \sigma ^ {-1} $ che \`e ancora una permutazione.
\newline
Avendo verificato l'esistenza di un prodotto associativo, un elemento identit\`a e dell'operazione inversa, l'insieme delle permutazioni forma un gruppo.
%
%
%
\newline\newline
A questi assiomi possiamo aggiungere un quinto: la \textit{commutativit\`a}, per cui il risultato dell'operazione di gruppo pu\`o dipendere dall'ordine degli operandi per cui $ a \bullet b = b \bullet a$ pu\`o non essere sempre vera. 
\newline
I gruppi per i quali vale la commutativit\`a sono detti \textit{Gruppi Abeliani}.
\newline\newline
%
%
%
\textbf{Anello}: Un anello \`e un insieme sul quale sono definite due operazioni binarie (struttura algebrica) che, per estensione dei numeri interi $\mathbb{Z}$, vengono chiamate \textit{somma} ($+$) e \textit{prodotto} ($\cdot$). L'anello \`e un gruppo abeliano sull'operazione binaria della (somma); consideriamo un esempio sui numeri interi: l'insieme \`e $\mathbb{Z}$, l'operazione \`e la somma "+" ed il gruppo viene quindi denotato con $(\mathbb{Z}, +)$. L'operazione + combina due elementi qualsiasi e ne forma un terzo, \`e un'operazione associativa, l'elemento identit\`a \`e lo zero, ogni intero $n \in \mathbb{Z}$ ha il suo elemento opposto $-n$, infine \`e un'operazione commutativa dato che vale $m + n = n + m, \forall (n, m) \in \mathbb{Z}$.
\newline
I numeri interi $\mathbb{Z}$ presi in modulo $n$ costituiscono il gruppo $\mathbb{Z}/ n \mathbb{Z}$ degli \textit{Interi in modulo $n$}. Tali gruppi $G$, detti ciclici, possono esser generati da un unico elemento, consistono in un insieme di elementi con una singola operazione associativa invertibile e contiene un elemento $g$, detto \textit{Generatore} del gruppo, tale che ogni altro elemento del gruppo pu\`o essere ottenuto applicando ripetitivamente l'operazione di gruppo o la sua inversa a tale elemento $g$. Ogni elemento del gruppo pu\`o esser scritto come potenza di $g$ nelle notazioni della moltiplicazione o dell'addizione.
\newline
Per un anello pu\`o valere l'uguaglianza $0 = 1$ qualora sia formato da un solo elemento; viene perci\`o detto \textit{anello banale}.
\newline\newline
%
%
%
\textbf{Caratteristica di un Anello}: Dato un anello $R$ ed un elemento $a$ la sua caratteristica $char(R)$ \`e definita come \begin{center} $n  \in \mathbb{N}$, $n \ne 0 $, $char(R) = \begin{cases} \min (n) \mid \underbrace{a+a+\cdots+a}_\text{n volte} = 0 & \mbox{se }\exists n\\
0 & \mbox{altrimenti}
\end{cases}$
\end{center}
Se tale minimo $n$ non esiste allora si assume che la caratteristica sia 0 per definizione.
Per un anello la caratteristica diventa il pi\`u piccolo $n$ tale che valga $a+a+\cdots+a{{=}}0$ per ogni suo elemento; la si pu\`o definire come il \textit{minimo comune multiplo} delle caratteristiche di tutti i suoi elementi. 
Tutti gli anelli con caratteristica $0$ sono \textit{infiniti}.
\newline
L'unico anello con caratteristica $1$ \`e l'anello banale.
\newline
Tutti gli altri anelli con caratteristica maggiore di 1 sono costituiti da un numero finito di elementi. Suddivisioni dei numeri interi del tipo $\mathbb{Z}/n\mathbb{Z}$ forniscono anelli modulari con elementi in numero finito ed hanno caratteristica $char(R) = n$.\newline
I campi $\mathbb{Q}$, $\mathbb{R}$ e $\mathbb{C}$ hanno caratteristica $0$.
\newline\newline
%
%
%
\textbf{Ideale}: un ideale \`e un sottoinsieme di un anello chiuso rispetto alla somma interna e rispetto al prodotto con qualsiasi elemento dell'anello.
Dato un anello $R$, diciamo che l'ideale $P$ di $R$ \`e \textbf{primo} se ha le seguenti propriet\`a:
\begin{enumerate}
    \item $P$ \`e un sottoinsieme proprio di $R$,
    \item dati gli elementi $(a, b) \in R \mid ab \in P$ allora almeno uno dei due elementi appartiene a $P$.
\end{enumerate}
Quanto detto generalizza la seguente propriet\`a dei numeri primi: "se $p$ \`e un numero primo e tale $p$ divide il prodotto di due interi $ab$, allora $p$ divide $a$ o $p$ divide $b$. Possiamo quindi dire che un numero intero positivo $n$ \`e un numero primo se e solo se $n\mathbb{Z}$ \`e un \textit{Ideale Primo} in $\mathbb{Z}$. Considerato l'anello $R$ come insieme dei numeri interi $\mathbb{Z}$ (vale quindi $R = \mathbb{Z}$), l'insieme di tutti i numeri \textit{pari} costituisce un Primo Ideale.
Un ulteriore esempio: scelto l'anello $\mathbb{C}[x, y]$ di polinomi in due variabili a coefficienti complessi, l'ideale generato dal polinomio $y^2 - x^3 -x -1$ \`e un Ideale Primo.
\newline\newline
%
%
%
\textbf{Spazio Topologico - Definizione tramite Insiemi Aperti}: data la coppia $(X, T)$ dove $X$ \`e un insieme e $T$ \`e una collezione di sottoinsiemi di $X$, gli spazi topologici devono soddisfare le seguenti:
\begin{enumerate}
    \item l'insieme vuoto ed $X$ si trovano in $T$,
    \item l'unione di qualsiasi collezione di insiemi di $T$ deve trovarsi ancora in $T$,
    \item l'intersezione di due qualsiasi insiemi di $T$ deve trovarsi ancora in $T$.
\end{enumerate}
Gli insiemi in $T$ sono \textit{insiemi aperti}.
\newline
Uno spazio topologico non si basa sulla distanza tra due punti ma sulla forma dello spazio stesso.
\newline\newline
%
%
%
\textbf{Spazio Proiettivo}: \`e ottenuto da uno spazio euclideo aggiungendo i \textit{punti all'infinito}. A seconda dello spazio euclideo considerato, ad esempio una retta o un piano, si vengono a definire, nello spazio proiettivo, una retta proiettiva o un piano proiettivo. Tale spazio viene identificato dal simbolo $\mathbb{P}$.
\newline
Estendere il piano Euclideo a quello proiettivo significa:
\begin{enumerate}
    \item per ogni classe di rette parallele, aggiungere un singolo nuovo punto. Tale punto \`e considerato come punto di incontro per ogni retta della stessa classe. Classi di diverse rette parallele avranno diversi punti. Questi punti sono chiamati \textit{punti all'infinito};
    \item aggiungere una nuova retta incidente su tutti e soli i punti all'infinito. Questa retta \`e chiamata \textbf{la} \textit{retta all'infinito}.
\end{enumerate}
Quanto appena detto esclude l'esistenza di rette parallele.
\newline
Lo spazio proiettivo $n$-dimensionale \`e l'unione di $\mathbb{R}^n$ e di tutti i suoi punti all'infinito. Tuttavia questa definizione pone i punti all'infinito come \textit{punti speciali}, per tal motivo una definizione pi\`u rigorosa \`e: lo spazio proiettivo $n$-dimensionale \`e definito come l'insieme delle rette $\mathbb{R}^ {n+1}$ passanti per l'origine.
\newline
Lo spazio proiettivo reale $\textbf{RP}^n$ o $\mathbb{P}_n(\mathbb{R})$ \`e lo spazio topologico di rette in $\mathbb{R}^ {n+1}$ passanti per l'origine.
\newline
Per uno spazio proiettivo parliamo di \textit{coordinate omogenee} o \textit{coordinate proiettive} tali per cui, dato il punto E$(x, y)$ nel piano Euclideo, le coordinate omogenee vengono rappresentate in forma E$(xZ, yZ, Z)$ dove $Z \in \mathbb{R} \backslash \{0\}$ \`e un numero reale diverso da zero. Il punto $(0, 0, 0)$ non esiste in quanto $Z$ deve essere diverso da $0$ e non esiste l'origine in uno spazio proiettivo. Diversamente da quanto accade per lo spazio Euclideo, due insiemi di coordinate omogenee rappresentano lo stesso punto se e solo se un insieme \`e ottenuto moltiplicando le coordinate per una costante diversa da 0. Ci\`o significa che, se prendiamo il punto sul piano cartesiano $(1, 2)$, questo pu\`o esser rappresentato dal punto omogeneo $(1, 2, 1)$ o $2, 4, 2)$.\\
Dal punto omogeneo $(X, Y, Z)$ otteniamo il punto cartesiano $(X/Z, Y/Z)$. Un caso degenere \`e per $Z=0$ per cui il punto rappresentato nello spazio proiettivo \`e il punto all'infinito. L'insieme di tutti i punti $(x, y, 0)$, ovvero di tutti i punti all'infinito, \`e la retta all'infinito.
\newline
Riguardo la notazione: per distinguere le coordinate omogenee da quelle cartesiane, spesso vengono usate notazioni differenti da $(x, y)$. A volte viene sostituita la virgola $,$ con il simbolo $:$, altre volte le parentesi tonde $()$ vengono sostituite da quadre $[]$ ed altre volte vengono usate entrambe le notazioni assieme. Esempio: coordinate cartesiane $(1, 4)$, omogenee $[1:4:Z]$.
\newline\newline
%
%
%
\textbf{Spazio Affine}: struttura geometrica che generalizza quelle propriet\`a dello spazio Euclideo che sono indipendenti dai concetti di distanza e di misura degli angoli, mantenendo solo le propriet\`a di parallelismo e rapporto di lunghezze per segmenti paralleli. In uno spazio affine non viene distinto un punto di "origine". Di conseguenza non possono esistere vettori con un'origine fissata e non possono essere univocamente associati ad un punto. In uno spazio affine esistono i vettori distanza (displacement vectors) anche chiamati \textit{traslazioni} tra due punti nello spazio. La dimensione di uno spazio affine \`e dunque pari alla dimensione dello spazio dei vettori di traslazione. Per uno spazio affine di dimensione uno $\mathbb{A}^1$ abbiamo una \textit{linea affine}, per uno spazio affine di dimensione due $\mathbb{A}^2$ abbiamo un \textit{piano affine}.
\newline
Gli spazi affini sono sottospazi degli spazi proiettivi: otteniamo un piano affine da un qualsiasi piano proiettivo rimuovendo la linea all'infinito ed i punti all'infinito. \`E quindi l'operazione inversa rispetto alla trasformazione dal piano euclideo a quello proiettivo. Similmente possiamo fare il passaggio inverso, ovvero costruire un piano proiettivo a partire da un piano affine aggiungendo la linea all'infinito e tutti i suoi punti.
%Se un piano affine contiene un numero finito di punti e se una retta del piano contiene $n$ punti allora:
%\begin{enumerate}
%    \item tutte le rette contengono $n$ punti,
%    \item ogni punto del piano \`e contenuto in $n+1$ rette,
%    \item esistono $n^2$ punti in totale,
%    \item esistono $n^2 + n = n(n+1)$ rette.
%\end{enumerate}
%Il numero $n$ viene detto \textit{ordine} del piano affine.
%\includegraphics[scale=0.3]{PianoAffine}
\newline\newline
%
%
%
\textbf{Variet\`a - \textit{Manifold}}: Una variet\`a algebrica $V$ \`e l'insieme delle soluzioni di un sistema di polinomi sui numeri reali o complessi. Tramite le variet\`a algebriche \`e possibile creare un legame tra l'algebra e la geometria in modo tale da formulare problemi geometrici in termini algebrici e viceversa. Possiamo dunque dire che un punto del piano appartiene alla curva $C$ se le sue coordinate soddisfano l'equazione polinomiale data la cui soluzione \`e data dalla variet\`a algebrica.
In una variet\`a algebrica $V$ chiamiamo il punto P \textit{singolare} se non pu\`o essere definita una tangente in quel punto. Per variet\`a definite sui reali il concetto di punto singolare generalizza il concetto di \textit{non derivabilit\`a}. Una variet\`a algebrica che non presenti punti singolari viene detta non singolare, \textit{liscia} o $\mathbb{C}^ {\infty}$ ed implica l'esistenza di derivate di ogni ordine.
Classi di variet\`a algebriche sono le curve algebriche piane, le quali includono: rette, circonferenze, parabole, ellissi, iperbole, curve cubiche come ad esempio le curve ellittiche.
\newline\newline
Definiamo una variet\`a affine $V$ su un campo chiuso $K$ come il luogo degli zeri in uno spazio affine $A^n$ $n$-dimensionale di una famiglia finita di polinomi in $n$ variabili con coefficienti in $K$ che generano un Ideale Primo. Alternativamente possiamo dire che, detta $V$ una variet\`a algebrica su un campo $K$, Diciamo che $V$ \`e una variet\`a affine se \`e data da un insieme di equazioni $f_j(x_1, \ldots , x_n) = 0$ $\forall j = 1, \ldots, m$ con coefficienti in $K$.
\newline
Esempio: chiamiamo $K$ un campo chiuso e $A^2$ lo spazio affine bidimensionale su $K$. I polinomi nell'anello $K[x, y]$ possono esser visti come funzioni a valori complessi su $A^2$ valutati nei punti di $A^2$. Il sottoinsieme S dell'anello che contiene un singolo elemento $f(x, y) = x + y - 1$. Il luogo degli zeri di $f(x, y)$ \`e l'insieme dei punti in $A^2$ sui quali la funzione si annulla, ovvero l'insieme di tutte le coppie di numeri complessi $(x, y)$ tali che $y = 1 - x$ (linea). L'insieme $V(f) = {(x, 1-x) \in C^2}$ \`e un sottoinsieme di $A^2$ ed un insieme algebrico. Tale insieme $V$ non \`e vuoto, \`e irriducibile dato che non pu\`o essere scritto come unione di due sottoinsieme algebrici e quindi \`e una variet\`a algebrica affine.
%
%
%
\newline\newline
Parliamo di variet\`a proiettiva quando ci riferiamo ad una sottovariet\`a di uno spazio proiettivo. Tale variet\`a \`e il luogo degli zeri di un insieme di polinomi omogenei che generano un Primo Ideale.
\newline
Una curva proiettiva piana \`e il luogo degli zeri di un insieme di polinomi omogenei irriducibili in tre incognite. La curva proiettiva $y^2 = x^3 -x$ nello spazio affine di dimensione 2, ha associata l'equazione polinomiale omogenea cubica $y^2z = x^3 - xz^2$ che definisce una curva nello spazio proiettivo $\mathbb{P}^2$ come \textit{Curva Ellittica}. 
\newline
Introduciamo infine la formula \textit{genere-grado} che lega il grado $d$ di una curva piana $C \subset \mathbb{P}^2$ con il genere geometrico $g = \frac{1}{2} (d-1)(d-2)$.
\newline
Si deduce che per la curva sopra descritta abbiamo grado $d = 3$ e quindi genere $g = 1$.
\newline\newline
%
%
%
\textbf{Curva piana cubica}: curva algebrica piana $C$ di grado 3 definita da un'equazione $F(x, y, z) = 0$ applicata alle coordinate omogenee di un piano proiettivo. 
%Combinazioni lineari non nulle di monomi del terzo grado sono: $x^3, y^3, z^3, x^2y, x^2z, y^2x, y^2z, z^2x, z^2y, xyz$, per un totale di 10 combinazioni. Per questo motivo le curve cubiche formano uno spazio proiettivo di dimensione 9 per ogni campo $K$ dato.
%Una curva $C$ passante per un punto $P$ comporta che ogni $P$ imponga una singola condizione lineare sulla $F$; per tal motivo \`e possibile trovare curve cubiche su qualsiasi 9 punti dati.
\newline
Una curva cubica pu\`o presentare un punto singolare, in questo caso assume la parametrizzazione in termini di una retta proiettiva. In caso di una curva liscia (priva di punti singolari) la curva ha 9 punti di inflessione (punti di flesso) su un campo chiuso, ad esempio il campo $\mathbb{C}$ dei numeri complessi. 
%Di questi punti, solo 3 possono essere reali e gli altri 6 non possono esser mostrati nel piano proiettivo disegnando la curva. 
%Una propriet\`a di questi punti \`e che ogni retta passante per due di questi contiene esattamente tre punti di inflessione.
\newline\newline
%
%
%
\textbf{Punto Razionale}: un punto $K$-razionale \`e un punto $P(x, y)$ di una curva algebrica $f(x, y)=0$ dove ciascuna delle sue coordinate $x$ e $y$ appartengono al campo $K$. In questo senso un punto razionale $(x, y)$ del campo $\mathbb{Q}$ soddisfa l'equazione data $f$ con $x$ e $y$ entrambi numeri appartenenti al campo $\mathbb{Q}$. 
\newline
%Definita, inoltre, una variet\`a affine $V$ data dall'insieme delle equazioni $f_j(x_1, \ldots , x_n) = 0$ $\forall j = 1, \ldots, m$ con coefficienti in $K$, il punto $K$-razionale di $V$ \`e una ennupla ordinata $(x_1, \ldots , x_n)$ di elementi del campo $K$ la quale \`e soluzione simultanea di tutto l'insieme di equazioni $f_j$.
Un punto razionale pu\`o anche essere il punto all'infinito.
\newline\newline
%
%
%
%\textbf{Funzione Abeliana}: funzione $f(u_1, ... u_p)$ analitica uniforme nelle variabili p, periodica, dipendente da tutte e p le sue variabili, meromorfa.
\newline\newline
%
%
%
%\textbf{Funzione Meromorfa}: \`e una funzione Olomorfa (Derivabile nel campo complesso per ogni direzione) in un sottodominio D del campo complesso $\mathbb{C}$, ad esclusione di un numero finito di punti detti Poli della funzione\newline\newline
%
%
%
%
%
%
%
%
%
%
\chapter{Curve Ellittiche}
Una curva ellittica pu\`o esser definita come una curva Algebrica in due variabili, di grado 3, genere 1, avente un punto $K$-razionale. Il punto $K$-razionale pu\`o essere il punto all'infinito $O$ ed il campo $K$ \`e solitamente il campo dei numeri reali $\mathbb{C}$, dei reali $\mathbb{R}$, dei razionali $\mathbb{Q}$ o un campo finito. Se la caratteristica $char(K)$ del campo $K$ \`e diversa 2 e da 3 possiamo scrivere la formula della nella forma proiettiva estesa di Tate-Weierstrass :
\begin{gather}
Y^{2}Z + a_1XYZ + a_3YZ^2 =X^3 + a_2X^2Z + a_4XZ^2 + a_6X^3
\end{gather}
La forma affine estesa \`e ottenuta impostando $Z=1$: 
\begin{gather}
Y^{2} + a_1XY + a_3Y =X^3 + a_2X^2 + a_4X + a_6
\end{gather}
Tramite due cambi di variabili otteniamo la forma breve della curva: lavoriamo dapprima sul membro di sinistra ed applichiamo la trasformazione $y = Y - \frac{a_1X + a_3}{2}$
\begin{align*}
&Y^{2} + a_1XY + a_3Y  
\\ 
&= \left ( y - \frac{a_1}{2}X - \frac{a_3}{2} \right )^2 + a_1X \left ( y - \frac{a_1}{2}X - \frac{a_3}{2} \right ) + a_3 \left ( y - \frac{a_1}{2}X - \frac{a_3}{2} \right ) 
\\
%= y^2 -a_1Xy -a_3y + \frac{{a_1}^2}{4}X^2 +\frac{a_1a_3}{2}X + \frac{{a_3}^2}{4} + a_1Xy - \frac{{a_1}^2}{2}X^2 - \frac{a_1a_3}{2}X + a_3y - \frac{a_1a_3}{2}X - \frac{{a_3}^2}{2} 
%\\
&= y^2 + \frac{a_1^2}{2}X^2 - \frac{a_1a_3}{2}X +\frac{a_3^2}{2} 
\end{align*}
\\
Portiamo quindi i termini in $X$ al membro destro dell'equazione e semplifichiamo i termini simili
\begin{align*}
y^2 &= - \left ( \frac{{a_1}^2}{2}X^2 - \frac{a_1a_3}{2}X +\frac{{a_3}^2}{2} \right ) + X^3 + a_2X^2 + a_4X + a_6
\\
&= X^3 + AX^2 + BX + C 
\end{align*}
dove abbiamo $A = a_2 - \frac{{a_1}^2}{2}$, $B = a_4 + \frac{a_1a_3}{2}$ e $C = a_6 - \frac{{a_3}^2}{2}$.
Applichiamo quindi il secondo cambio di variabile $x = X - \frac{A}{3}$, il secondo membro diventa:
\begin{align*}
&X^3 + AX^2 + BX + C 
\\
&=\left(x- \frac{A}{3} \right )^3 +A\left ( x - \frac{A}{3} \right )^2 + B\left (x - \frac{A}{3} \right )^3 +C
\\
&= x^3-Ax^2 + \frac{A^2}{3}x - \frac{A^3}{27} + Ax^2 + - \frac{A^3}{9} -\frac{2}{3}A^2x + Bx - \frac{AB}{3} +C
\\
&= x^3 + (B-A^2)x + \frac{2A^3-9AB+27C}{27}
\\
&= x^3 + ax+b
\end{align*}
Abbiamo quindi ottenuto la forma breve
\begin{gather}
y^2 = x^3 + ax+b
\end{gather}
Tuttavia per renderla una vera curva ellittica dobbiamo imporre che sia liscia, ovvero non singolare, per cui non devono esistere radici multiple. Possiamo aggiungere questa condizione dicendo che il determinante dell'espressione $x^3 + ax+b$ deve essere diverso da zero, ovvero $\Delta = -4a^3 - 27b^2 \ne 0$. L'aggiunta di questa condizione ci porta alla
\begin{center}
\textit{Equazione di Weierstrass}:
$\begin{cases}
y^2 = x^3 + ax+b\\
4a^3 \ne 27b^2\end{cases}$
\end{center}
\textbf{Dimostrazione $4a^3 \ne 27b^2$}: La condizione da imporre per una curva liscia \`e la derivabilit\`a in ogni punto della curva, va quindi rispettata la relazione $\frac{d}{dx}C(x) \ne 0 \forall x \in C$ dove $C$ \`e l'equazione della nostra curva. Applicando la derivata:\\
$\frac{d}{dx}y^2 = \frac{d}{dx}(x^3 +Ax+B)$\\
$0 = 3x^2 +A$\\
$x^2 = -\frac{A}{3}$\\
Applichiamo ora la $x^2$ al termine destro dell'equazione breve moltiplicato per $x$, ovvero:\\
$x^4 + Ax^2+Bx = 0$ \\
$(-\frac{A}{3})^2 + A(-\frac{A}{3})+Bx = 0$\\
$(\frac{A^2}{9})^2 - \frac{A^2}{3}+Bx = 0$\\
$x = \frac{2A^2}{9B}$\\
Il valore trovato pu\`o ora essere applicato alla derivata in $dx$ trovata prima:\\
$0 = 3x^2 +A$\\
$0 = 3(\frac{2A^2}{9B})^2 +A = \frac{4A^4}{27B^2} +A $\\
Dividendo per $A$ otteniamo finalmente:\\
$0 = \frac{4A^3}{27B^2} +1 = 4A^3+27B$\\
Pi\`u chiaramente abbiamo che la derivata si annulla per $4A^3+27B = 0$. Questo dimostra la tesi iniziale.\\
\\
La curva ellittica nel piano proiettivo ha un singolo punto all'infinito $O$ avente coordinate omogenee $[0:1:0]$, tale punto costituisce l'elemento identit\`a del gruppo. Inoltre la curva \`e simmetrica rispetto l'asse $x$, per ogni punto $P$ possiamo trovare $-P$ sulla curva al punto opposto rispetto l'asse di simmetria. Per convenzione il punto all'infinito $-O$ viene considerato semplicemente come $O$.\\
\`E possibile affermare che le curve ellittiche formano un Gruppo date le seguenti affermazioni:
\begin{itemize}
    \item gli elementi del gruppo sono tutti e soli i punti della curva,
    \item l'elemento identit\`a \`e il punto all'infinito $O = 0$,
    \item l'elemento inverso del punto $P$ \`e il punto simmetrico rispetto all'asse $x$,
    \item la legge di gruppo "somma" \`e data dalla legge "dati tre punti $A$, $B$ e $C$ allineati, diversi da $0$, la loro somma \`e $A+B+C=0$.
\end{itemize}
Quest'ultima affermazione resta valida dati tre punti allineati qualsiasi, senza pesare sul loro ordine. Questo ci porta ad affermare che se $A$, $B$ e $C$ sono tre punti allineati allora:\\
$A+(B+C)=B+(A+C)=C+(A+B)=C+(B+A)=\cdots = 0$\\
dimostrando intuitivamente che l'operatore \textit{somma} "$+$" \`e sia associativo che commutativo. Il gruppo definito da una curva ellittica \`e quindi un gruppo abeliano.
\begin{center}
\includegraphics[scale=0.8]{Curvabase2}
\\
Tipica rappresentazione di una curva ellittica
%\includegraphics[scale=0.7]{EC16a+21mod23}
\\
%Rappresentazione della curva $y^2 = x^3 + 16x + 21$
\end{center}
\begin{center}
\textbf{Point Addition}
\end{center}
Con il termine "Point Addition" si fa riferimento alla somma di due punti sulla curva costituisce la legge di gruppo per le curve ellittiche e viene definita come segue: 
\\
Dati due punti distinti $K$-razionali $A$ e $B$ sulla curva possiamo definire in modo univoco un terzo punto $C(x_C, y_C)$ dato dall'intersezione della retta passante per $A$ e $B$ e la curva. Il punto opposto $-C(x_C, -y_C)$ rappresenta la somma dei due punti: $A + B = B+ A = -C$. Il risultato dell'equazione $A+B+C = 0$ \`e il punto all'infinito. Il motivo di tale risultato \`e che la curva ellittica ha un singolo punto all'infinito il quale \`e un punto di flesso. Si pu\`o aggiungere che la linea all'infinito incontrer\`a una sola volta la curva, precisamente nel punto all'infinito, quindi tale punto determina un'intersezione di molteplicit\`a tre.
\begin{center}
\includegraphics[scale=0.7]{PointAdditionA+Bbn}
\end{center}
Esistono per\`o dei casi particolari:
\begin{enumerate}
    \item Somma di un punto $P$ ed il punto all'infinito $O$. 
    \\
    Ricordando che il punto $O$ viene trattato come elemento identit\`a del gruppo possiamo scrivere $P+O = O+P = P$,
    \item Somma di due punti simmetrici (opposti) tra loro $P + (-P)$. 
    \\
    Tramite il metodo del Point Addition si costruisce una retta per i due punti che risulta parallela all'asse y che andr\`a ad intersecare il punto all'infinito $O$. Definiamo quindi $P + (-P) = O$,
    \item Somma di un punto $P$ e se stesso. 
    \\
    Non avendo due punti distinti non \`e possibile tracciare la retta per i due punti; in questo caso si traccia la retta tangente alla curva nel punto $P$, viene quindi trovato un altro punto $-Q$ ed il suo simmetrico \`e il risultato cercato: $P + P = Q$,
    \item Caso particolare: sommiamo $P+P$ ma tale punto \`e un flesso per la curva. 
    \\
    In un punto di flesso la concavit\`a della curva cambia ed in ogni curva ellittica esistono esattamente 9 punti di flesso. In questo caso consideriamo il secondo punto $Q =P$ e la somma finale viene ad essere $-P$, l'opposto del punto iniziale.
\end{enumerate}
Il calcolo del punto $C = A + B$ \`e dato per costruzione geometrica della retta $r$ passante per i punti $A(x_A, y_A)$ e $B(x_B, y_B)$. Detto $m$ il coefficiente angolare della retta $r_{AB}$, le coordinate del punto $C(x_C, y_C) = -(P+Q)$ saranno date dalle seguenti formule:
\begin{align*}
\begin{cases}
m = \frac{y_A - y_B}{x_A - x_B}\\
x_C = m^2 - (x_A + x_B)\\
y_C = m(x_A - x_C)-y_A
\end{cases}
\end{align*}
%Aggiungi C = A + A e formule corrette
Si noti che nel caso in cui $x_A = x_B$ si ottiene uno zero al denominatore, $m$ porterebbe quindi la tangente ad essere parallela all'asse y ed il punto individuato dalla formula sarebbe il punto all'infinito $O$ per cui otterremmo $C = O$.
\\
Come gi\`a visto, se vale $x_A = x_B$ abbiamo dei casi particolari per cui se $y_A = -y_B$ allora i due punti sono simmetrici e la loro somma \`e il punto all'infinito, se invece abbiamo $y_A = y_B = 0$ abbiamo che il punto $A$ ed il punto $B$ coincidono e si trovano all'estremit\`a sinistra della curva. La loro somma \`e ancora il punto all'infinito. Tuttavia esiste un altro caso particolare in cui si verifica $x_A = x_B$ ma $y_A = y_B \ne 0$, tale caso \`e detto \textit{Point Doubling}.
\\
\\
\begin{center}
\textbf{Point Doubling}
\end{center}
L'operazione del Point Doubling permette di sommare un punto a se stesso, se la sua ordinata $y$ \`e diversa da zero, ed ottenere un secondo punto sulla curva. Dato un punto $A(x_A, y_A)$ con $y_A \ne 0$ sommare $A+A$ equivale a trovare il punto $B = -(A+A) = -2A$. Quella che nel Point Addition era una retta per due punti diventa ora la tangente alla curva $y^2 = x^3 + ax +b$ nel punto $A$. La motivazione che ci porta ad usare la retta tangente alla curva viene dalla considerazione che presi due punti $P$ e $Q$ della curva, man mano che $Q$ si avvicina a $P$ pi\`u la curva tra i due punti tende a diventare la tangente. Le equazioni che permettono il calcolo del punto $B$ sono:
\begin{align*}
\begin{cases}
m = \frac{3{x_A}^2 + a}{2y_A}\\
x_C = m^2 - 2x_A\\
y_C = m(x_A - x_C)-y_A
\end{cases}
\end{align*}
Si noti come il coefficiente angolare $m$ dipenda non solo dalle coordinate di $A$ ma anche dal coefficiente $a$ presente nell'equazione della curva. Tale formulazione discende dalla derivata prima dell'equazione della curva per cui si ha:
$\frac{d}{dx}y^2 = \frac{d}{dx}x^3 + ax + b$\\
$2y\frac{d}{dx} = 3x^2 + a$\\
$m = \frac{d}{dx} = \frac{3x^2 + a}{2y}$
\\
\\
Quanto detto finora si applica a curve ellittiche nel campo dei numeri reali $\mathbb{R}$, tuttavia possiamo estendere il discorso ai numeri razionali $\mathbb{Q}$ in quanto definiti sul campo dei reali. La legge di gruppo (Point Addition) tra punti con coordinate reali si applica anche al campo $\mathbb{Q}$; le formule usate mostrano che la somma di due punti $A, B \in \mathbb{Q}$ comporta $C = A + B, \to C \in \mathbb{Q}$ dato che la retta $r_{AB}$ \`e a coefficienti razionali. In tal modo si dimostra che l'insieme di punti razionali della curva forma un sottogruppo del gruppo a punti reali. Essendo quest'ultimo un gruppo abeliano, tale sar\`a anche il sottogruppo dei numeri razionali; si conclude quindi che vale $C = A+B = B+A$.
\\
\\
Studiare una curva ellittica su di un campo finito (ovvero il numero dei suoi elementi \`e finito) porta la curva ad una "scomposizione" in punti, anch'essi in numero finito, perdendo la tipica forma vista negli esempi precedenti. L'equazione che governa la curva cambia leggermente: consideriamo $(x, y) \in \mathbb{Z}/p\mathbb{Z}$ con $p$ un numero primo\\
$\begin{cases}
y^2 = x^3 + ax+b (mod p)\\
4a^3 \ne 27b^2  (mod p)
\end{cases}$
in unione al punto all'infinito.\\
\\
Il numero di elementi di una curva ellittica in un campo finito \`e difficile da determinare ma possiamo limitare in un intervallo tale cardinalit\`a grazie al teorema di Hasse sulle curve ellittiche: detti $K$ un campo finito con un $q$ elementi, $E$ una curva ellittica definita su tale $K$ e $cardE_K$ la cardinalit\`a (numero di elementi) della curva, possiamo dire che
\begin{align*}
    \mid cardE_K - (q+1) \mid \le 2 \sqrt{q}
\end{align*}
che ci porta a delimitare la cardinalit\`a tra i valori 
\begin{align*}
    q+1-2 \sqrt {q} \le cardE_K \le q+1+2 \sqrt {q}
\end{align*}
Consideriamo degli esempi: per un campo $K$ di $q=4$ elementi abbiamo una curva $E$ con $1 \le cardE_K \le 9$, $q=16 \to 9 \le cardE_K \le 25$, $q=64 \to 33 \le cardE_K \le 97$, $q=256 \to 129 \le cardE_K \le 385$. \`E importante notare che il teorema usato resituisce la cardinalit\`a della curva includendo il punto all'infinito.
Questi risultati ci portano ad affermare che il numero di elementi della curva \`e vicino al numero di elementi del campo. 
\textit{Considerazioni personali}: per numeri $q$ quadrati perfetti si pu\`o semplificare la formula di Hasse in $ \frac{1}{2}q +1 \le cardE_K \le \frac{3}{2}q +1$. Questa formula \`e stata trovata dall'osservazione dei risultati degli esempi precedenti e messa a confronto con numeri "non quadrati perfetti" come ad esempio un numero primo. La formula mostrata fornisce una stima coerente con quella di Hasse per "quadrati" ma fornisce una stima peggiorativa in tutti gli altri casi. Esempio: $q=160$, per Hasse abbiamo $136 \le cardE_K \le 186$, per la formula ideata abbiamo $81 \le cardE_K \le 241$. 
\\
L'insieme di punti di $E$ per il campo dato $K_p$ \`e un gruppo abeliano finito, \`e inoltre un gruppo ciclico o il risultato di un prodotto tra due gruppi ciclici. Un esempio \`e dato dalla curva $y^2 = x^3 - x$ sul campo $K_{71} = \mathbb{Z}/2\mathbb{Z}\times\mathbb{Z}/36\mathbb{Z}$ che comprende un totale di 72 punti di cui 71 sono punti affini, incluso il punto $(0,0)$ ed uno \`e il punto all'infinito.
Numerosi algoritmi permettono il calcolo preciso del numero di punti per una curva ellittica su un campo finito; ad esempio l'algoritmo di Schoof parte dal teorema di Hasse, definisce $t = q+1-cardE_K$ e si procede nel calcolare la cardinalit\`a di $t\%N$ (operazione modulo), dove $N > 4\sqrt{q}$, tramite il teorema cinese del resto. La complessit\`a di questo algoritmo \`e pari a $O(n^{5+o(1)})$.
%Altri algoritmi
%1 - https://en.wikipedia.org/wiki/Counting_points_on_elliptic_curves
%2 - https://en.wikipedia.org/wiki/Elliptic_curve#Elliptic_curves_over_finite_fields
%
\\
\\
\begin{center}
\includegraphics[scale=0.8]{ECmodp}\\
Le immagini mostrano la stessa curva $y^2=x^3-7x+10$ mod $(p)$ con \\
$p = \{19, 97, 127, 487\}$.\\
Si noti inoltre come l'asse di simmetria non sia pi\`u per $y=0$ ma diventa $y=p\big / 2$
\end{center}
Risulta evidente che la curva abbia perso la sua tipica forma e di come sia composta di soli punti sparsi sul piano $xy$. Le operazioni studiate precedentemente non cambiano nonostante ci troviamo in presenza di un campo modulare. Le rette usate per la Point Addition e la Point Doubling non sono cambiate concettualmente, ma una differenza grafica notevole \`e che trovandoci in un campo modulare $p$ sono permesse coordinate con valori compresi tra $0$ e $p-1$, quei punti che escono fuori da questo intervallo vengono riportati in modulo $p$ sul grafico, lo stesso concetto si applica alla retta per cui viene ripetuta ciclicamente:
\begin{center}
\includegraphics[scale=0.65]{ECmodRetta}\\
Curva $y^2=x^3-x+3$ mod$(127)$\\
Punti $P = (16, 20)$, $Q=(41, 120)$, $R = -(P+Q) = (86, 81)$\\
Cardinalit\`a: 111 elementi
\end{center}

Quanto velocemente riusciamo a calcolare il punto $nP$? Poniamo il caso di dover calcolare $n=100$, tramite ripetute Point Addition abbiamo il punto $P$ di innesco e dobbiamo computare i successivi 99 punti per trovare $100P$ per un totale di $n-1$ addizioni. Velocizzare questo risultato diventa davvero facile se pensiamo di usare un insieme di Point Addition (Add) e Point Doubling (Double): $n=1 \to A$, Double $\to n=2$, Add $\to n=3$, Double $\to n=6$, Double $\to n=12$, Double $\to n=24$, Add $\to n=25$, Double $\to n=50$ ed infine Double $n=100$ per un totale di 8 operazioni (6 Point Doubling e 2 Point Addition). Abbiamo ridotto notevolmente il numero di operazioni necessarie per raggiungere il coefficiente $n$. 
\\
L'uso di una notazione base $B$ aiuta a trovare un algoritmo efficiente per questi calcoli; partiamo dal considerare $100_{10} = 1100100_2 = d_6d_5d_4d_3d_2d_1d_0$ ed usiamo il seguente algoritmo: scorriamo il numero in base $B=2$ da sinistra verso destra ed usiamo l'operazione Point Doubling, nel caso di dover calcolare un $1$ facciamo seguire una Point Addition. 
\\
\\
Passo 0: $n=1_2 \to nP = P$. Il primo numero da trovare sar\`a sempre 1\\
Passo 1: Double $n \to 2n = 10_2$, dato che $d_5 = 1$ dobbiamo proseguire con una Add;\\
Passo 2: Add    $2 \to 3 = 11_2$. Procediamo quindi al calcolo degli altri $d_i$;\\
Passo 3: Double $3 \to 6 = 110_2$\\
Passo 4: Double $6 \to 12 = 1100_2$\\
Passo 5: Double $12 \to 24 = 11000_2$, dobbiamo avere $d_2 = 1$ quindi...\\
Passo 6: Add    $24 \to 25 = 11001_2$\\
Passo 7: Double $25 \to 50 = 110010_2$\\
Passo 8: Double $50 \to 100 = 1100100_2$\\
\\
Usando le operazioni indicate troviamo il punto cercato $100P$. Confrontiamo ora i costi computazionali dei due metodi visti. 
\\
Il primo consiste nel sommare $n-1$ volte lo stesso punto per arrivare al nostro $nP$; prendiamo il numero $n-1$ ed espresso in termini binari, base 2, vediamo che $n_2$ \`e formato da $k$ cifre $d_kd_{k-1} \ldots d_1d_0$. Il costo di $n-1$ addizioni \`e pari a $O(n-1)$ ed abbiamo visto che non \`e affatto un buon risultato. 
\\
Il secondo metodo, detto Double and Add, necessita di sole $O(log n)$ operazioni, dandoci un risultato nettamente migliore.
%
%
%
%
\chapter{Sottogruppi}
%
Data la curva $E: y^2 = x^3 +2x +2$ mod$(17)$ e dato il punto $A(5, 1)$, usiamo la tecnica del Point Doubling per trovare il punto $2A$.
I dati che ci servono sono $a = 2$, coefficiente di $x$ nella curva, e le coordinate del punto $A$; passiamo alle formule:
\\
\\
$m = \frac{3{x_A}^2 + a}{2y_A}= \frac{3 \cdot 5^2 + 2}{2 \cdot 1} = \frac{9}{2} \text{ mod}17= 9 \cdot 9 \text{ mod}17 = 13\text{ mod}17\\
x_C = m^2 - 2x_A= 13^2 - (2 \cdot 5) = 169\text{ mod}17 = 6 \text{ mod}17\\
y_C = m(x_A - x_C)-y_A= m(5 - 6)-1 = -14\text{ mod}17 = 3\text{ mod}17
$
\\
\\
Il punto trovato ha coordinate $2A = (5, 1)+(5, 1) = (6, 3)$.
\\
\begin{center}
\includegraphics[scale=0.6]{PointDoublingMOD17}
\end{center}
Come si nota dal grafico, i punti rappresentati sono in numero limitato. La cardinalit\`a secondo il teorema di Hasse \`e compresa nell'intervallo $[10, 26]$, se ci riferiamo alla figura e contiamo i punti rappresentati notiamo che questi sono in totale 19, compreso il punto all'infinito. Il risultato teorico rispetta il valore effettivo trovato.
\\
\`E possibile fare un'ulteriore verifica del risultato contando i punti tramite una Point Addition nel seguente modo: sappiamo che partendo dal punto $A(5, 1)$ abbiamo trovato $2A = (6, 3)$; possiamo quindi sommare i due punti trovati ed ottenere $3A = 2A + A = (10, 6)$. Procedendo analogamente troviamo $4A = (3, 1)$, $5A = (9, 16)$, $6A = (16, 13)$, $7A = (0, 6)$, $8A = (13, 7)$, $9A = (7, 6)$, $10A = (7, 11)$, $11A = (13, 10)$, $12A = (0, 11)$, $13A = (16, 4)$,  $14A = (9, 1)$, $15A = (3, 16)$, $16A = (10, 11)$, $17A = (6, 14)$, $18A = (5, 16)$, $19A = O$.
\\
Il punto $19A$ rappresenta il punto all'infinito.
Alcune considerazioni sui risultati ottenuti:
\begin{enumerate}
    
    \item Ancora una volta sono state rispettate le aspettative teoriche sulla cardinalit\`a della curva, abbiamo infatti trovato 19 punti distinti sulla curva.
    
    \item $19A$ \`e il punto all'infinito perch\`e corrisponde alla Point Addition $(5, 1) + (5, 16)$. L'ascissa \`e $5$ per entrambi i punti e l'ordinata \`e diversa da zero, i due punti sono quindi distinti sulla curva e, come gi\`a detto durante lo studio della Point Addition, il risultato deve essere il punto $O$ all'infinito.
    
    \item La somma di $A + 18A$ \`e il punto $19A = O$ ma lo stesso risultato lo si pu\`o ottenere sommando altri due punti come $2A + 17A$, $3A + 16A$, $4A + 15A$ e cos\`i via ottenendo sempre lo stesso risultato (lo si nota facilmente considerando che le ascisse delle coppie di punti dati sono le stesse).
    
    \item \textit{Considerazioni personali}: il calcolo dei 19 punti ha portato alla luce una ricorsivit\`a nei valori delle ascisse, i primi 9 punti hanno ascissa nella sequenza $5, 6, 10, 3, 9, 16, 0, 13, 7$ mentre i successivi 9 hanno ascissa in sequenza invertita. Due punti successivi tra loro presentano la stessa ascissa: $9A$ e $10A$, la loro somma \`e $19A = O$. 
    %Partendo da questo risultato \`e forse possibile trovare "facilmente" la cardinalit\`a di una curva? 
    \\
    \`E possibile affermare "calcolati due punti $nA$ e $(n+1)A$ di una curva $E$ su un campo finito $K$ tramite la legge di gruppo Point Addition, se vale $x_n = x_{n+1}$ allora il $nA + (n+1)A$ corrisponde al punto all'infinito e la somma delle molteplicit\`a dei due punti costituisce la Cardinalit\`a, per cui $cardE_K = n+(n+1) = 2n+1$"?
    \\
    La risposta \`e in generale "no". Quanto appena affermato non vale all'interno di gruppi "grandi"; prendiamo ad esempio la curva $y^2 = x^3 + 2x+3$ nel campo $\mathbb{Z}/97\mathbb{Z}$, con punto di partenza $P(3, 6)$. La curva ha in totale 100 punti ma il punto $P$ da noi considerato genera un sottogruppo, per il quale viene detto \textit{Generatore}, con meno di 100 punti. I punti sono $P(3, 6), 2P(80, 10), 3P(80, 87), 4P(3, 91), 5P = O$ da cui deduciamo che la cardinalit\`a di questo sottogruppo \`e 5, vale inoltre $2P+3P = 5P \to card(E_K) = 5$ in quanto $2P$ e $3P$ sono punti successivi tra loro con medesima ascissa; la ricorsivit\`a delle ascisse vista precedentemente si mantiene valida nel sottogruppo per cui possiamo scrivere ogni punto $kP$ come $(k$ mod $card(E_K))P$.\\
    Abbiamo infine provato che diversi punti di innesco generano diversi sottogruppi, ognuno con una sua cardinalit\`a $card(S_E) \le card(E)$ (nota: viene rispettata l'uguaglianza tra i due termini quando non esistono sottogruppi, esattamente come accadeva nel primo esempio). 
    
\end{enumerate}

L'ordine del punto $P$ \`e analogo al concetto di caratteristica $char(K)$ di un campo $K$ ovvero corrisponde al pi\`u piccolo numero intero positivo tale che $nP = 0$. L'ordine di $P$ \`e direttamente collegato alla caratteristica della curva ellitta per mezzo del \textit{Teorema di Lagrange} che afferma "la cardinalit\`a di un sottogruppo $H$ del gruppo $G$ \`e un divisore intero della cardinalit\`a di $G$"; in altri termini se il gruppo $G$ ha cardinalit\`a $N$ ed uno dei suoi sottogruppi ha cardinalit\`a $n$ allora $n$ \`e un divisore intero di $N$. Detto ci\`o possiamo costruirci un metodo per trovare la cardinalit\`a di un sottogruppo $H$ dato il punto generatore $P$: calcoliamo $N = card(E_K)$ tramite l'algoritmo di Schoof, prendiamo tutti i divisori interi di $N$ trovando quindi $n_1$, $n_2 \ldots$. Il pi\`u piccolo $n_x$ per cui verifichiamo $n_xP = 0$ corrisponde alla cardinalit\`a del sottogruppo.
\\
Esempio: la curva $y^2 = x^3-x+3$ mod$(37)$ ha cardinalit\`a $N=42$ per cui i suoi sottogruppi possono avere cardinalit\`a pari a $n= \{1, 2, 3, 6, 7, 14, 21\}$. Dato il punto $P=(2, 3)$ troviamo che la cardinalit\`a del sottogruppo $H_P$ \`e pari a 7 perch\`e $7P = 0$. Ovviamente anche $n = 14$, $n = 21$ ed $n=42$ portano a rispettare l'equazione $n_xP=0$ ma solo il divisore pi\`u piccolo corrisponde alla cardinalit\`a del sottogruppo.
\\
\textbf{Ulteriore esempio}: curva $E: y^2 = x^3 -x+1$ mod$(29)$ ha cardinalit\`a $N=37$, un numero primo. I divisori di un numero primo sono solo $1$ e se stesso per cui la cardinalit\`a dei sottogruppi $H_G$ pu\`o essere o $1$ o $37$, tuttavia non possiamo accettare una cardinalit\`a $n=1$ in quanto ricordiamo che tale sottogruppo comprenderebbe solo il punto all'infinito. Deduciamo quindi che $n = N = 37$ \`e l'unico valore accettabile per la cardinalit\`a dei sottogruppi $H_G$ e che \textbf{esiste un solo sottogruppo $H$} corrispondente al gruppo generatore $G$ e comprendente tutti i punti della curva.
\\
Sempre come conseguenza del teorema di Lagrange possiamo individuare un numero $h= N \big / n, \mid h \in \mathbb{Z}$ che chiamiamo \textbf{cofattore del sottogruppo}. \`E importante notare che per ogni punto della curva viene verificata la seguente uguaglianza $NP=0$ poich\`e $N$ sar\`a sempre un multiplo di $n$, euesto ci permette di scrivere $n(hP)=0$. Supponiamo che $n$ sia un numero primo, l'equazione lascia quindi pensare che il punto $G = hP$ sia un generatore di un sottogruppo di cardinalit\`a $n$.
\\
Trovare un sottogruppo con la cardinalit\`a pi\`u grande possibile:
\begin{enumerate}
    \item Data la curva ellittica $E$ calcolare $N = Schoof(E)$ per ottenerne la cardinalit\`a;
    \item Calcoliamo il divisore \textbf{primo} pi\`u grande di $N$;
    \item Calcoliamo $h = N \big / n$;
    \item Continuiamo a scegliere un punto casuale $P$ sulla curva finch\`e $G = hP = 0$. Quando troviamo un valore per cui $G \ne 0$ allora possiamo dire che $P$ \`e un generatore del sottogruppo $H$ di cardinalit\`a $n$ e cofattore $h$.
\end{enumerate}

\chapter{Logaritmo Discreto}
In matematica non \`e solitamente difficile trovare un numero $k$ data l'equazione $Q = kP$, nelle curve ellittiche il discorso cambia notevolmente. Trovare il coefficiente $k$ viene considerato un problema Difficile per la matematica, il nome con cui ci riferiamo a questo \`e \textit{Logaritmo Discreto}.
\\
Questo problema \`e analogo al problema del logaritmo discreto usato con sistemi di crittografia quali Digital Signature Algorithm (DSA), lo scambio di chiavi Diffie-Hellman (D-H) e l'algoritmo ElGamal. Questi algoritmi sono caratterizzati dal trovare il numero $k$ dall'espressione esponenziale con modulo: $b = a^k$ mod$p$. 
\\
L'aggettivo "ciclico" viene dato per via degli insiemi finiti o sottogruppi ciclici usati dagli algoritmi. Notiamo per\`o che non \`e presente alcun logaritmo per le curve ellittiche: la denominazione proviene dalla conformazione con l'espressione appena vista. L'espressione per le curve ellittiche presenta una difficolt\`a computazionale ben superiore ai problemi logartmici discreti analoghi, si \`e pensato quindi di usare questo problema come base per un sistema di crittografia chiamato Crittografia Ellittica.




%
%
%
%
%
%
%
%
%
%
%
%\bibliografia{tesi}
%
%
%
%
%
%
%
%
%
%
%\appendice
%\chapter{prima appendice}
%
%
%
%
%
%
%
%
%
%
%\chapter{seconda appendice}
%
%
%
%
%
%
%
%
%
%
\end{document}
